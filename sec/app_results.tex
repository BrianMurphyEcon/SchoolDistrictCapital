\subsection{Minority groups without own language\label{sec:app:esti:minoownlang}}

Materials from this example appendix comes from \href{https://linkinghub.elsevier.com/retrieve/pii/S0305750X21003491}{Hannum and Wang (2022)}. 

There is significant linguistic and cultural diversity among and within ethnic groups \autocite{harrell_linguistics_1993, dwyer_texture_1998}. While most minority groups in China have at least one spoken language, some do not \autocite{wang_chinas_2015}. The linguistic distinctions might capture not only differences in language usage but also other aspects of culture, all of which might relate to the effects of closure. For robustness, in this section, we re-estimate our main educational attainment and Mandarin ability specifications from Tables \ref{tab:app:one} and
\ref{tab:app:two} but now exclude minority individuals that do not have their own language. Remaining minority individuals are potentially more sharply distinct from Han individuals.

In our first robustness exercise, we drop cases in which minority respondents are reported as not having a minority language. In our second robustness exercise, we drop cases for minority groups in which over 50\% of respondents are reported as not having a minority language. Hui individuals account for 59 and 95 percent of the excluded individuals under the first and second exclusion scenarios, respectively. In both exercises, the Han group stays the same. Attainment results under the first and second exclusion scenario are shown in columns 1 to 3 and 4 to 6 of Table \ref{tab:app:one}, respectively.\footnote{In an additional set of exercises, we use all data and estimate heterogeneous effects for Han individuals, minorities with own languages, and minorities without own languages. Estimates for minorities with and without own language have similar directions, but estimates for minorities without own language have standard errors that are more than twice as large due to the limited sample size. Results are available upon request from the authors, but not presented here for conciseness.}

\blindtext

\subsection{Additional Mandarin ability regressions with non-respondents\label{sec:app:esti:mandwithna}}

In this section, we test the robustness of not including non-response to the language ability questions. In Table \ref{tab:app:one}, we add non-respondents to the written and spoken Mandarin ability regressions from Table \ref{tab:app:two}. As shown in Table \ref{tab:app:one}, non-respondents to the Mandarin ability questions, who account for 10 percent of the analytic sample, have substantially lower enrollment and attainment statistics compared to individuals who were reported by the household interviewee as having strong Mandarin ability. Given this, for the regressions in Table \ref{tab:app:two}, we include non-respondents in the same group as individuals who have no, simple, or good Mandarin ability. We find that the results shown in Table \ref{tab:app:one} is largely invariant to the addition of non-respondents.

\blindtext

