\subsection{Using RIF to Decompose Changes in Distributional Statistics beyond the Mean\label{RIF}}

Materials from this example appendix comes from \href{https://fanwangecon.github.io/assets/BhalotraFernandezWangMexicoFLFP.pdf}{Bhalotra, Fernandez, and Wang (2023)}. 

\citet{Firpoetal07, Firpoetal09} allow extending the traditional Oaxaca-Blinder decomposition to distributional statistics beyond the mean. This is achieved through the use of influence functions (IF). To see this, let $q_{\tau}(F_{W})$ be $\tau$th quantile of the distribution of wages, expressed in terms of the cumulative distribution $F_{W}(w)$. Define the following mixture distribution:

\begin{equation}
\label{eq=mixture_function}
G_{W,\epsilon}=(1-\epsilon)F_{W}+\epsilon H_{W} \;\;\;\; for \;\;\;\; 0 \leq \epsilon \leq 1,
\end{equation}

\noindent where $H_{W}$ is some perturbation distribution that only puts mass at the value $w$. In that case, $G_{W,\epsilon}$ is a distribution where, with probability $(1-\epsilon)$, the observation is generated by $F_{W}$, and with probability $\epsilon$, the observation takes the arbitrary value of the perturbation distribution. By definition, the influence function corresponds to:  

\begin{equation}
IF(w;q_{\tau}, F_{W}) = lim_{\epsilon \rightarrow 0} \;\; \frac{q_{\tau}(G_{W,\epsilon})-q_{\tau}(F_{W})}{\epsilon},
\end{equation}

\noindent where the expression is analogous to the directional derivative of $q_{\tau}$ in the direction of $H_{W}$. Analytical expressions for influence functions have been derived for many distributional statistics.\footnote{\citet{Essama11} provides a comprehensive list of influence functions for different distributional statistics.} The influence function in the case of the $\tau$th quantile takes the form: 

\begin{equation}\label{eq=ifq}
IF(w; q_{\tau}, F_{W}) = \frac{\tau - \mathbbm{1}[w \leq q_{\tau}]} {f_{W}(q_{\tau})},
\end{equation}

\noindent where $\mathbbm{1}[\cdot]$ is an indicator function and $f_{W}$ is the PDF.\footnote{Note that the influence function in this case depends on the density. In order to obtain the empirical density the authors propose non-parametric kernel density estimation.} Using some of the properties of influence functions, a direct link with the traditional Oaxaca-Blinder approach can be established. In particular, a property that is shared by influence functions is that, by definition, the expectation is equal to zero:

\begin{equation}
\int_{-\infty}^{+\infty} IF(w;q_{\tau}, F_{W}) d F(w)=0.
\end{equation}

\citet{Firpoetal09} propose a simple modification in which the quantile is added back to the influence function, resulting in what the authors call the Recentered Influence Function (RIF):

\begin{equation}
RIF(w;q_{\tau}, F_{W})=q_{\tau}+IF(w;q_{\tau}, F_{W}).
\end{equation}

\noindent The importance of this transformation lies in the fact that the expectation of the RIF is precisely the quantile $q_{\tau}$. With this result, \citet{Firpoetal09} show that we can model the conditional expectation of the RIF as a linear function of the explanatory variables:

\begin{equation}
\label{eqtext=rif_linear}
E[RIF(w_{t};q_{\tau}, F_{W,t}|X_{t})] = X_{t}'\gamma_{t}.
\end{equation}

Moreover, if we apply the law of iterated expectations to Equation \eqref{eqtext=rif_linear}, the end result is an expression that directly relates the impact of changes in the expected values of the covariates on the unconditional quantile $q_{\tau}$. Note that this result is all that is required to extend the Oaxaca-Blinder decomposition to quantiles, since the basic components of the method are all present in Equation (\ref{eqtext=rif_linear}).

Estimation of Equation (\ref{eqtext=rif_linear}) can be done by OLS, and only requires replacing the dependent variable, $\log w_{t}$ in the original wage setting model with the RIF of the quantile $q_{\tau}$. The interpretation of the estimates $\widehat{\gamma}_{t}$ can be thought of as the effect of a small change in the distribution of $X$ on $q_{\tau}$, or as linear approximation of the effect of large changes of $X$ on $q_{\tau}$ \citep{Firpoetal07}.